\chapter{Smoothed Particle Hydrodynamics}
\label{chap:SPH}

In the present work Smoothed Particle Hydrodynamics (SPH), a particle based
framework is used to study the fluid motion, using Navier-stokes
equation as basis for fluid motion.


\section{Introduction to SPH}


It is invented by \cite{lucy1977numerical} and
\cite{gingold1977smoothed} independently to model astrophysical
simulations. The basic idea of SPH is representing a field variable
A from known values of A through interpolation integrals in the given
domain. This could be written in terms of Dirac delta function as follows

\begin{equation}
  \label{eq:dirac_repr}
  A(\boldsymbol{x}) = \int_{\Omega}\> A(\boldsymbol{x}') \> \delta(\boldsymbol{x} - \boldsymbol{x}') \> d\boldsymbol{x}'
\end{equation}

Where x is the point of interest in the domain, $\omega$ is volume of
the domain and $d\boldsymbol{x}'$ is the differential volume element.
To prove that the above equation \eqref{eq:dirac_repr} interpolates the
field variable correctly, since Dirac delta function is zero except at
$\boldsymbol{x}$, and at $\boldsymbol{x}$ A is constant, so one can take
A out of the integral, which gives us the value at $\boldsymbol{x}$.


\begin{align*}
  \label{eq:dirac_derivation}
  A(\boldsymbol{x}) &= \int_{\Omega}\> A(\boldsymbol{x}') \> \delta(\boldsymbol{x} - \boldsymbol{x}') \> d\boldsymbol{x}'\\
  A(\boldsymbol{x}) &= A(\boldsymbol{x}) \int_{\Omega}\> \> \delta(\boldsymbol{x} - \boldsymbol{x}') \> d\boldsymbol{x}'\\
  A(\boldsymbol{x}) &= A(\boldsymbol{x}) \\
\end{align*}

A can be approximated by replacing the delta function with a smoothing
kernel W, with smoothing length h. Resulting interpolation equation is given by

\begin{equation}
  \label{eq:kernel_repr}
  A(\boldsymbol{x}) = \int_{\Omega}\> A(\boldsymbol{x}') \> W(\boldsymbol{x} - \boldsymbol{x}', h)  \> d\boldsymbol{x}'
\end{equation}

The smoothing function W is usually chosen to be as even function
satisfying few properties.

The first condition is Unity condition:

\begin{equation}
  \label{eq:unity}
  \int_{\Omega}\> W(\boldsymbol{x} - \boldsymbol{x}', h)  \> d\boldsymbol{x}' = 1
\end{equation}

The second condition is kernel function W behaving like delta function
as the influence length h goes to 0. Since delta function has infinite
value at a single point and zero at rest, similarly when the influence
length of the kernel function goes to zero, it should behave like
delta function.

\begin{equation}
  \label{eq:kernel_delta}
  \lim_{h \to 0} W(\boldsymbol{x} - \boldsymbol{x}', h) = \delta(\boldsymbol{x} - \boldsymbol{x}', h)
\end{equation}

And the third condition is having a compact support. As discussed before
that the kernel function has a support, and out of that support the value of
it has to be zero. It is given by \eqref{eq:compact_support}

\begin{equation}
  \label{eq:compact_support}
  W(\boldsymbol{x} - \boldsymbol{x}', h) = 0 \>\>\>\>    when  \>\>\> |\boldsymbol{x} - \boldsymbol{x}'| > kh
\end{equation}

Where k depends on the smoothing kernel chosen.

Changing continuous form of \eqref{eq:kernel_repr} to particle form using
the particles in the domain of interest.

Assume that the domain is discretized into N particles with mass $m_i$ and volume
$dv_i$. We have

\begin{equation}
  \label{eq:mass_repr}
  m_j = dv_i \> \rho_i
\end{equation}

Using \eqref{eq:mass_repr}, \eqref{eq:kernel_repr} is converted to
discrete form as follows

\begin{equation}
  \label{eq:discrete_form}
  A_i = \sum\> \frac{m_j}{\rho_j} A_j\> W(\boldsymbol{x}_i - \boldsymbol{x}_j, h)
\end{equation}

Where $A_i$ is the value of the field property of particle i,
similarly for $A_j$.

Similarly derivative of a field variable can be written as

\begin{equation}
  \label{eq:discrete_derivative}
  A_i = \sum\> \frac{m_j}{\rho_j} A_j\> \nabla W(\boldsymbol{x}_i - \boldsymbol{x}_j, h)
\end{equation}


\section{Fluid equations of motion}

Navier-Stokes equations are a set of partial differential equations
that are used to describe the motion of fluids. These equations are
used to model different types of scenarios like : air flow through
cars, aeroplanes, and etcetera.  For fluid phase the following
equations are used.

The equation of motion is given by


\begin{equation}
  \label{eq:momentum_eq}
  \rho \>\frac{d\textbf{v}}{dt} = - \nabla p + \mu\> \nabla^2\textbf{v} + \textbf{f}
\end{equation}

Where \textbf{v} is flow velocity, $\rho$ is density, p is the
pressure and \textbf{f} is body force. In order to compute the right
hand side of \eqref{eq:momentum_eq} we need information, such as
density of the fluid, and pressure at every time step, which could be
obtained using other equations.


In Lagrangian point of view  \eqref{eq:momentum_eq} is written as

\begin{equation}
  \label{eq:momentum_eq}
  \rho \>\frac{\partial\textbf{v}}{\partial t} = - \nabla p + \mu\> \nabla^2\textbf{v} + \textbf{f}
\end{equation}

This could be written in terms of forces acting on a particle with mass m as

\begin{equation*}
  \label{eq:forces_eq}
  m_i \>\frac{\partial\textbf{v}_i}{\partial t} = -V_i \nabla p_i + V_i \mu\> \nabla^2\textbf{v} + V_i \textbf{f}
\end{equation*}

The right hand side of \eqref{eq:forces_eq} are the forces acting on
particle i due to pressure of neighbouring particles, shear force due
to relative motion between neighbouring particles, and the third term
is the body force. Here there is a clear advantage of particle approach,
suppose in future if one has to apply surface tension between particles,
or force due to solid body immersed in fluid, or due to porous media,
one could simply add the force term to this equation \eqref{eq:forces_eq}.


After computing the forces on each particle, respective accelerations are calculated
from the following

\begin{equation*}
  \label{eq:total_force}
    \textbf{F}_i = \textbf{F}_{i}^{Pressure} + \textbf{F}_{i}^{viscous} + \textbf{F}_{i}^{additional}
\end{equation*}

\begin{equation}
  \label{eq:acceleration}
    \textbf{a}_i = \frac{\textbf{F}_i}{m_i}
\end{equation}

By using appropriate integration schemes like RK2, Euler, Prediction
correction algorithms the system is moved forward by computing
velocity and position of the particles at every time step by using
acceleration computed from \eqref{eq:acceleration}


As discussed previously in order to compute the pressure force we need
additional equations to solve for pressure computations. Since in the
present work we deal with WCSPH, pressure is directly related to
density with Tait equation. From this one could infer that we need to
solve for density at every time step. There are several ways to
compute density in SPH, two methods are discussed in the next section.


\section{Density computation}
\label{sec:density_comp}

Density of a fluid particle in SPH can be computed using many approaches, like

\begin{itemize}
\item Summation density
\item Continuity equation
\end{itemize}

SPH way of density computing is using summation density. It is derived from basic
function interpolation in SPH. It could be derived as follows.

From \eqref{eq:summation_mass_rep} we have,

\begin{equation*}
  \label{eq:summation_mass_rep}
  A_i = \sum\> \frac{m_j}{\rho_j} A_j\> W(i,j)
\end{equation*}

Where, $A_i$ is the property we are interested in. Here it is density,
by substituting $A_i$ with $\rho_i$ we have

\begin{align}
  \label{eq:summation_mass_rep}
    \rho_i &= \sum\> \frac{m_j}{\rho_j} \rho_j\> W(i,j) \nonumber \\
    \rho_i &= \sum\> m_j \> W(i,j)
\end{align}


By using \eqref{eq:summation_mass_rep} one gets lower density values
at the free surface of the fluid. This is due to lack of particle
support. In order to prevent such thing, Shepard filter is applied.

In the second approach from continuity equation the rate of change of
density is calculated, through which the density at a time is
computed.

\begin{align}
  \label{eq:continuity}
    \frac{d\rho}{dt} = -\rho \nabla * \textbf{v}
\end{align}

\eqref{eq:continuity} is discretized in terms of particles and written
in summation form as follows

\begin{align}
  \label{eq:sph_continuity}
    \frac{d\rho_a}{dt} = \sum_{b}m_b \textbf{v}_{ab} * \nabla_a W_{ab}
\end{align}

Once density of the field is calculated, the corresponding pressure is computed using
Tait equations, which is described in next section.


\section{Pressure computation}

Since the fluid is slightly compressible, one can relate pressure with density
of fluid. The reason for considering the fluid to be slightly compressible is because,
if the flow in incompressible, one has to solve for pressure Poisson equation in order to
compute the pressure of the fluid particle, which is quite computationally expensive.
By considering the flow as compressible such time consuming computation could be reduced.

Pressure and density could be related with two equations, they are, the ideal gas equation
and Tait equation. In the present work Tait equation of state is used. Tait equation of
state is given as

\begin{align}
  \label{eq:pressure}
  p_i = \frac{\rho_0 c_s^2 }{\gamma} \bigg(\bigg(\frac{\rho}{\rho_0}\bigg)^\gamma - 1 \bigg)
\end{align}

where c_s is the maximum speed of sound inside the fluid, $\rho_0$ is
the rest density of the fluid and gamma is the polytropic constant.
\eqref{eq:pressure} is extended to handle low density situations as in
free surface flows and near particle leaking. In low density regions
where density of the fluid is less than the fluid density we have
negative pressure which leads to cohesion force rather than repulsive,
which is not entertained. In order to prevent such attraction hg
correction is applied to the equation \eqref{eq:pressure} The corrected
pressure equation is termed to be HG correction.





In SPH, the fluid is discretized into particles, contrast to euerian
grid based scemes. The physical properties of the fluid are obtained
at the particle, where as in eulerian frameworks, the properties are
calcualted at locations.


\section{Smoothing kernels}
\label{sec:sk}


In the present work cubic b-splpine kernel function is used
for averaging the neighbourhood particle contribution. The kernel
function can be written as

\[
  W_{cspline}(r, h) =
  \begin{cases}
    \frac{6 \> r^3}{h^3} - \frac{6 \> r^3}{h^2}, & 0 <= r <= \frac{h}{2}\\
    2 * \big(1 - \frac{r}{h}\big)^3, & \frac{h}{2} <= r <= h\\
    0,                                          & \text{otherwise}
  \end{cases}
\]


\section{Approximating fluid equation of motion with SPH}
\label{sec:fesph}

Navier-Stokes equations are a set of partial differential equations
that are used to describe the motion of fluids. These equations are
used to model different types of scenarios like : air flow through
cars, aeroplanes, and etcetera.  For fluid phase the following
equations are used.

The equatin of motion of a incompressible fluid in lagrangian view
point can be written as

\begin{equation}
  \label{eq:momentum_eq}
  \rho\> \frac{d\textbf{v}}{dt} = -\nabla p + \mu\> \nabla^2\textbf{v} + \textbf{f}
\end{equation}

Multiplying both sides by volume, we get
\begin{equation}
  \label{eq:momentum_eq}
  m\>\frac{d\textbf{v}}{dt} = - V\> \nabla p + V\> \mu\> \nabla^2\textbf{v} + V\> \textbf{f}
\end{equation}

Using SPH the forces on a given particle can be easily computed. Then
these ordinary differential equations can be solved using Euler or RK2
or any other integrating scheme.

\section{Density computation}
\label{sec:density_comp}

Density of a fluid particle in SPH can be computed using many approaches, like

\begin{itemize}
\item Summation density
\item Continuity equation
\end{itemize}

SPH way of density computing is using summation density. It is derived from basic
function interpolation in SPH. It could be derived as follows.

From \eqref{eq:summation_mass_rep} we have,

\begin{equation*}
  \label{eq:summation_mass_rep}
  A_i = \sum\> \frac{m_j}{\rho_j} A_j\> W(i,j)
\end{equation*}

Where, $A_i$ is the property we are interested in. Here it is density,
by substituting $A_i$ with $\rho_i$ we have

\begin{align}
  \label{eq:summation_mass_rep}
    \rho_i &= \sum\> \frac{m_j}{\rho_j} \rho_j\> W(i,j) \nonumber \\
    \rho_i &= \sum\> m_j \> W(i,j)
\end{align}


By using \eqref{eq:summation_mass_rep} one gets lower density values
at the free surface of the fluid. This is due to lack of particle
support. In order to prevent such thing, Shepard filter is applied.

In the second approach from continuity equation the rate of change of
density is calculated, through which the density at a time is
computed.

\begin{align}
  \label{eq:continuity}
    \frac{d\rho}{dt} = -\rho \nabla * \textbf{v}
\end{align}

\eqref{eq:continuity} is discretized in terms of particles and written
in summation form as follows

\begin{align}
  \label{eq:sph_continuity}
    \frac{d\rho_a}{dt} = \sum_{b}m_b \textbf{v}_{ab} * \nabla_a W_{ab}
\end{align}

In the present work WCSPH scheme is used. In WCSPH scheme the flow is
considered to be weakly compressible, which means there could be a variation
in the density of the fluid flow. And the pressure is directly related to density
of the particle by Tait's equation. Once the pressure of each particle is computed, the
resulting forces on a particle due to its neighbours is calculated by the following equation.

\begin{align}
  \label{eq:sph_momentum}
    \frac{d\textbf{v}_a}{dt} = \textbf{g}_a  -\sum_{b}m_b \Bigg(\frac{p_a}{\rho^2_a} + \frac{p_b}{\rho^2_b}\Bigg) \nabla_a W_{ab}
\end{align}

Where $p_a$ is the pressure of the

% Pressure equation:
% ----------------------------------------------
% ----------------------------------------------
% The relationship between pressure and density of a particle is given by

% $$ p_a = B \Bigg[\Bigg(\frac{\rho_a}{\rho_0}\Bigg)^\gamma - 1 \Bigg] $$

% LocalWords:  Tait polytropic Navier eq dirac repr kh discretized dv
% LocalWords:  dt RK WCSPH
