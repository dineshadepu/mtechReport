\chapter{Smoothed Particle Hydrodynamics}
\label{chap:SPH}

In the present work Smoothed Particle Hydrodynamics (SPH), a particle based
framework is used to study the fluid motion, using Naview-stokes
equation as basis for fluid motion.

In SPH, the fluid is discretized into particles, contrast to euerian
grid based scemes. The physical properties of the fluid are obtained
at the particle, where as in eulerian frameworks, the properties are
calcualted at locations.

Further, physical quantities of a particle are computed by weighted
average of neighbouring particles using appropriate kernels. In SPH,
the integral representation of a field variable A at location $x_i$ is
defined as

\begin{equation}
  \label{eq:integral_rep}
  A(x_i) = \int\> A(x_j) \>W(x_i - x_j, h)
\end{equation}

Where W is a kernel function. $dx_{j}$ is the volume of the particle
situated at $x_j$. Here W is like a weighted factor for the
contributions from the neighbourhood interpolation points.

Solving \eqref{eq:integral_rep} numerically, needs to discretize the
domain into particles. Now changing the integral representation to
summation form interms of the discretized points, gives us

\begin{equation}
  \label{eq:summation_rep}
  A_i = \sum\> A_j\> W(i,j)\>\> V_j
\end{equation}

Here $V_j$ is the volume of particle j, and $A_j$ is the interpolated
property of particle j.

\eqref{eq:summation_rep} can also be written interms of mass of the particle j
as

\begin{equation}
  \label{eq:summation_mass_rep}
  A_i = \sum\> \frac{m_j}{\rho_j} A_j\> W(i,j)
\end{equation}

Similarly the derivative of a field property can be written, interms
of derivative of kernel function, rather than the derivative of field
property as follows


\section{Smoothing kernels}
\label{sec:sk}


In the present work cubic b-splpine kernel function is used
for averaging the neighbourhood particle contribution. The kernel
function can be written as
