\chapter{Smoothed Particle Hydrodynamics}
\label{chap:SPH}

In the present work Smoothed Particle Hydrodynamics (SPH), a particle based
framework is used to study the fluid motion, using Navier-stokes
equation as basis for fluid motion.

In SPH, the fluid is discretized into particles, contrast to euerian
grid based scemes. The physical properties of the fluid are obtained
at the particle, where as in eulerian frameworks, the properties are
calcualted at locations.

Further, physical quantities of a particle are computed by weighted
average of neighbouring particles using appropriate kernels. In SPH,
the integral representation of a field variable A at location $x_i$ is
defined as

\begin{equation}
  \label{eq:integral_rep}
  A(x_i) = \int\> A(x_j) \>W(x_i - x_j, h)
\end{equation}

Where W is a kernel function. $dx_{j}$ is the volume of the particle
situated at $x_j$. Here W is like a weighted factor for the
contributions from the neighbourhood interpolation points.

Solving \eqref{eq:integral_rep} numerically, needs to discretize the
domain into particles. Now changing the integral representation to
summation form interms of the discretized points, gives us

\begin{equation}
  \label{eq:summation_rep}
  A_i = \sum\> A_j\> W(i,j)\>\> V_j
\end{equation}

Here $V_j$ is the volume of particle j, and $A_j$ is the interpolated
property of particle j.

\eqref{eq:summation_rep} can also be written interms of mass of the particle j
as

\begin{equation}
  \label{eq:summation_mass_rep}
  A_i = \sum\> \frac{m_j}{\rho_j} A_j\> W(i,j)
\end{equation}

Similarly the derivative of a field property can be written, interms
of derivative of kernel function, rather than the derivative of field
property as follows


\section{Smoothing kernels}
\label{sec:sk}


In the present work cubic b-splpine kernel function is used
for averaging the neighbourhood particle contribution. The kernel
function can be written as

\[
  W_{cspline}(r, h) =
  \begin{cases}
    \frac{6 \> r^3}{h^3} - \frac{6 \> r^3}{h^2}, & 0 <= r <= \frac{h}{2}\\
    2 * \big(1 - \frac{r}{h}\big)^3, & \frac{h}{2} <= r <= h\\
    0,                                          & \text{otherwise}
  \end{cases}
\]


\section{Approximating fluid equation of motion with SPH}
\label{sec:fesph}

Navier-Stokes equations are a set of partial differential equations
that are used to describe the motion of fluids. These equations are
used to model different types of scenarios like : air flow through
cars, aeroplanes, and etcetera.  For fluid phase the following
equations are used.

The equatin of motion of a incompressible fluid in lagrangian view
point can be written as

\begin{equation}
  \label{eq:momentum_eq}
  \rho\> \frac{d\textbf{v}}{dt} = -\nabla p + \mu\> \nabla^2\textbf{v} + \textbf{f}
\end{equation}

Multiplying both sides by volume, we get
\begin{equation}
  \label{eq:momentum_eq}
  m\>\frac{d\textbf{v}}{dt} = - V\> \nabla p + V\> \mu\> \nabla^2\textbf{v} + V\> \textbf{f}
\end{equation}

Using SPH the forces on a given particle can be easily computed. Then
these ordinary differential equations can be solved using Euler or RK2
or any other integrating scheme.

\section{Density computation}
\label{sec:density_comp}

Density of a fluid particle in SPH can be computed using many approaches, like

\begin{itemize}
\item Summation density
\item Continuity equation
\end{itemize}

SPH way of density computing is using summation density. It is derived from basic
function interpolation in SPH. It could be derived as follows.

From \eqref{eq:summation_mass_rep} we have,

\begin{equation*}
  \label{eq:summation_mass_rep}
  A_i = \sum\> \frac{m_j}{\rho_j} A_j\> W(i,j)
\end{equation*}

Where, $A_i$ is the property we are interested in. Here it is density,
by substituting $A_i$ with $\rho_i$ we have

\begin{align}
  \label{eq:summation_mass_rep}
    \rho_i &= \sum\> \frac{m_j}{\rho_j} \rho_j\> W(i,j) \nonumber \\
    \rho_i &= \sum\> m_j \> W(i,j)
\end{align}



% $$ \frac{d\rho}{dt} = -\rho \nabla * \textbf{v}$$
% <br />

% Discretized equations:
% ----------------------
% $$ \frac{d\rho_a}{dt} = \sum_{b}m_b \textbf{v}_{ab} * \nabla_a W_{ab}$$
% <br />
% $$ \frac{d\textbf{v}_a}{dt} = \textbf{g}_a  -\sum_{b}m_b \Bigg(\frac{p_a}{\rho^2_a} + \frac{p_b}{\rho^2_b}\Bigg) \nabla_a W_{ab}$$

% Pressure equation:
% ----------------------------------------------
% ----------------------------------------------
% The relationship between pressure and density of a particle is given by

% $$ p_a = B \Bigg[\Bigg(\frac{\rho_a}{\rho_0}\Bigg)^\gamma - 1 \Bigg] $$
