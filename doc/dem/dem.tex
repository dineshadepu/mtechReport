\chapter{Discrete Element Method}


Discrete element method is a numerical method used to predict the
motion of large number of small particles physically correct. This is
done using conservative and non conservative forces between the
colliding particles, then integrating their motion in time (see figure
Draw a figure in inkscape showing the basic methodology of DEM)

This method has three parts
\begin{itemize}
\item Identify the particle
\item Apply the forces due to particles in contact
\item Integrate the position and velocity
\end{itemize}


\section{Forces}
\label{sec:forces}

As the first step, in order to integrate the particles, one needs to
know the force acting on that particle.

As the particles comes into contact, there are mainly two forces which
act on them.  One is conservative force aka spring force, and the
other is non conservative force, which damps down the energy of the
system.
% \begin{figure}[ht]
%   \centering
%   \includegraphics[]{figures/}
%   \caption{\label{fig:label} }
% \end{figure}


\subsection{Conservative force or Spring force}

As two particles come into contact, an equal and opposite repulsive
force will act on each particle depending on the overlap amount. In
DEM this force can be modelled using a linear or non linear model.


Let a solid sphere be freely falling under gravity onto a flat floor.
The trajectory of the sphere of the particle until it hits the floor
can be computed with basic kinematics.

As the sphere comes into contact of the flat floor, there will be a
repulsive force on it. In computing this force, a linear force model
assumes a linear relationship between the amount of overlap and the
repulsive force as follows


\begin{align}
  \label{eq:linearForce}
  \label{eq1}F \propto \delta_{ij}\\
  \label{eq2}F = k * \delta_{ij}
\end{align}


to be in between the floor and sphere. Here $\delta_{ij}$ is the
amount of overlap and k is the stiffness of the spring which is
computed using properties of both the materials in contact.





%%


%%% Local Variables:
%%% mode: latex
%%% TeX-master: "../mainrep"
%%% End:
