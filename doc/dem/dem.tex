\chapter{Discrete Element Method}

It is a numerical method introduced by Cundall and Strack \citep{CS79}
used to model the behaviour of large number of particles having finite
mass and radius which interact at their surfaces. The governing
equation of motion for such a particle can be written using Newton's
laws.

\begin{align}
  \label{eq:newton_equation_of_motion}
  \frac{d^2 \vec{x}}{dt^2} = \vec{F}(\vec{x}, \vec{v}, m)\\
  \frac{d^2 \vec{\omega}}{dt^2} = \vec{M}(\vec{x}, \vec{v}, m)
\end{align}

Where $\vec{F}$, $\vec{M}$ is the force and moment acting on the
particle. m, $\vec{v}$, $\vec{x}$ are the mass, position and velocity
of the particle.
The force $\vec{F}$ in equation\ref{eq:newton_equation_of_motion} is
the one to be computed during each time step. This force arises due to
collision of particles.

% \begin{figure}
%   \centering
%   \includegraphics[scale=0.5]{dem/two_pars_in_contact}
%   \caption{Two spherical particles in contact}
%   \label{fig:tspc}
% \end{figure}

Before studying the interaction of a particle with wall or any other
spherical particle, lets consider a spherical particle with mass m
freely falling under gravity. For such a model, the governing
differential equation can be written as

\begin{align}
  \label{eq:free_fall}
  \frac{d^2 \vec{x}}{dt^2} = \vec{g}\\
  \frac{d^2 \vec{\omega}}{dt^2} = 0
\end{align}

Such a differential equation \eqref{eq:free_fall} governs a scene like
in figure \eqref{fig:free_fall}

\begin{figure}
\centering
\begin{subfigure}{.5\textwidth}
  \centering
  \includegraphics[scale=0.3]{dem/free_fall}
  \caption{A freely falling Sphere}
  \label{fig:free_fall}
\end{subfigure}%
\begin{subfigure}{.5\textwidth}
  \centering
  \includegraphics[scale=0.3]{dem/bounce_back}
  \caption{A freely falling Sphere onto a wall}
  \label{fig:bounce_back}
\end{subfigure}
\caption{Behaviour of sphere under different conditions}
\label{fig:sphere_intro}
\end{figure}


In freely falling situation, the particle doesn't see any obstacles,
and follows a straight line path.

Now assume that there is a wall obstructing the motion of this
spherical ball.  As the ball hits the wall it has to bounce back
\eqref{fig:bounce_back}, because of the repulsion force from the
wall. There are two ways to model such a behaviour.

\begin{itemize}
\item Event driven method (hard particles)
\item Discrete element methods (soft particles)
\end{itemize}


Present work deals with Discrete element method, with a little introduction
to Event driven method.

\section{Discrete element method (DEM)}
\label{sec:edm}

In this method at every instant of time the ball is checked if it is
in contact with any other particle or wall. If it is, then the force
is computed from the relative positions of the overlapping
particles. By using such force($\vec{F}$) \eqref{eq:newton_equation_of_motion} is
integrated using Euler or RK2.

\subsection{Force calculation}
\label{sec:force}

\begin{figure}
  \centering
  \includegraphics[scale=0.5]{dem/collision}
  \caption{Two spherical particles in contact}
  \label{fig:tspc}
\end{figure}

\begin{equation}
  \label{eq:overlap}
  \delta = (r_i + r_j) - |\vec{X}_{i} - \vec{X}_{j}|
\end{equation}

Say two particles \eqref{fig:tspc} i and j with radii $r_{i}$ and
$r_{j}$, are in contact if $\delta$ \eqref{eq:overlap} is positive.
The particle i has linear velocity $\vec{V}_{i}$ and angular velocity
of $\vec{\omega}_{i}$ particle j has linear velocity $\vec{V}_{j}$ angular
velocity of $\vec{\omega}_{j}$.

The unit normal vector along particle i to particle j  is given by

\begin{equation}
  \label{eq:normal_unit}
  \vec{n}_{ij} = \frac{\vec{X}_{j} - \vec{X}_{i}}{|\vec{X}_{j} - \vec{X}_{i}|}
\end{equation}

The relative velocity of point of contact becomes

\begin{equation}
  \label{eq:relative}
  \vec{V}_{ij} = \vec{V}_{i} - \vec{V}_{j} +
  (r_{i} \vec{\omega}_{i} + r_{j} \vec{\omega}_{j}) \times \vec{n}_{ij}
\end{equation}

Therefore the normal and tangential components of contact velocity are

\begin{equation}
  \label{eq:normal_vel}
  \vec{V}_{nij} = \vec{V}_{ij} \cdot \vec{n}_{ij} \vec{n}_{ij}
\end{equation}

\begin{equation}
  \label{eq:tang_vel}
  \vec{V}_{tij} = \vec{V}_{ij} - \vec{V}_{ij} \cdot \vec{n}_{ij} \vec{n}_{ij}
\end{equation}

The tangent to the plane of contact is

\begin{equation}
  \label{eq:tang_unit}
  \vec{t}_{ij} = \frac{\vec{V}_{tij}}{|\vec{V}_{tij}|}
\end{equation}


Discrete element method is a soft sphere approach. In soft sphere
approach the overlap between two particles is described by a spring
and a dashpot in both normal and tangential direction. Generally, when
two particles collide, there will be a repulsion force on both the
particles. This repulsive force is represented by springs, whose
magnitude depends on the overlap amount. Similarly, when a collision
occurs, there will be damping of energy due to deformation of the body
and the friction force between the particles. This damping is
introduced by the dashpot between the particles. The respective spring
and damping coefficients depend on the coefficient of restitution
between the colliding particles and the material properties of the
colliding particles.



%%


%%% Local Variables:
%%% mode: latex
%%% TeX-master: "../mainrep"
%%% End:
