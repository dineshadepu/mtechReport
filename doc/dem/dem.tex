\chapter{Discrete Element Method}

It is a numerical method introduced by Cundall and Strack \citep{CS79}
used to model the behaviour of large number of particles having finite
mass and radius which interact at their surfaces. The governing
equation of motion for such a particle can be written using Newton's
laws.

\begin{align}
  \label{eq:newton_equation_of_motion}
  \frac{d^2 \vec{x}}{dt^2} = \vec{F}(\vec{x}, \vec{v}, m)\\
  \frac{d^2 \vec{\omega}}{dt^2} = \vec{M}(\vec{x}, \vec{v}, m)
\end{align}


Where $\vec{F}$, $\vec{M}$ is the force and moment acting on the
particle. m, $\vec{v}$, $\vec{x}$ are the mass, position and velocity
of the particle.

The force $\vec{F}$ in equation\ref{eq:newton_equation_of_motion} is
the one to be computed during each time step. This force arises due to
collision of particles.

\begin{figure}[!h]
\begin{center}
\includegraphics[width=0.9\textwidth]{dem/two_pars_in_contact}
\caption{TDMA channel Access Mechanism} \label{fig:tdma}
\end{center}
\end{figure}

% \begin{figure}
%   \centering
%   \includegraphics[scale=1.5]{two_pars_in_contact}
%   \caption{Two spherical particles in contact}
%   \label{fig:two spherical particles in contact}
% \end{figure}

Assume at a given instant two particles are in contact as in
% figure\ref{fig:two spherical particles in contact}.


Granular material consist of a large number of particles. There particles interact
via short range forces, i.e., only via mechanical contact.

The dynamics of a granular material is governed by Newton's equation of motion.
In modelling rigid body collision DEM can be used.

%%


%%% Local Variables:
%%% mode: latex
%%% TeX-master: "../mainrep"
%%% End:
