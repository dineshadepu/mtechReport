\chapter{Discrete Element Method}


Discrete element method is a numerical method used to predict the
motion of large number of small particles physically correct. This is
done using conservative and non conservative forces between the
colliding particles, then integrating their motion in time (see figure
Draw a figure in inkscape showing the basic methodology of DEM)

This method has three parts
\begin{itemize}
\item Identify the particle
\item Apply the forces due to particles in contact
\item Integrate the position and velocity
\end{itemize}


\section{Forces}
\label{sec:forces}

As the first step, in order to integrate the particles, one needs to
know the force acting on that particle.

As the particles comes into contact, there are mainly two forces which act on them.
One is conservative force aka spring force, and the other is non conservative force, which damps
down the energy of the system.
% \begin{figure}[ht]
%   \centering
%   \includegraphics[]{figures/}
%   \caption{\label{fig:label} }
% \end{figure}



%%


%%% Local Variables:
%%% mode: latex
%%% TeX-master: "../mainrep"
%%% End:
